\RequirePackage[hyphens]{url}

\documentclass[11pt,a4paper]{article}

\usepackage[utf8]{inputenc}
\usepackage[T1]{fontenc}
\usepackage{lmodern}
\usepackage{listings}
\usepackage{amsmath}
\allowdisplaybreaks
\usepackage{amsfonts}
\usepackage{amstext}
\usepackage{amssymb}
\usepackage{amsthm}
\usepackage{anyfontsize}
%\usepackage{enumitem}
\usepackage{pdfpages}
\usepackage{fourier}	% Style
\usepackage{bm}
\usepackage{epstopdf}
\usepackage{lipsum}
\usepackage{authblk}
\usepackage[top=2cm, bottom=3cm, left=2cm, right=2cm, scale=0.75]{geometry}	% Set the margins
\usepackage{fancyhdr}
\usepackage[letterspace=150]{microtype}
\usepackage{textcomp}
\usepackage{gensymb}
\usepackage{booktabs}
\usepackage{amsmath,etoolbox}
\usepackage{mathtools}
\usepackage{array}
\usepackage{anyfontsize}
\usepackage{enumerate}
\usepackage{graphicx}
\usepackage{epstopdf}
\usepackage{float}
\usepackage{subfig}
\usepackage[labelfont=bf,labelsep=period]{caption}
\usepackage{booktabs}
\usepackage{newunicodechar}
\usepackage{booktabs}
\usepackage{nicefrac}	% For diagonal fractions
\usepackage{bbm}

% Packages needed for tables
\usepackage{longtable}
\usepackage{multicol}
\usepackage{multirow}
\usepackage{array}

\PassOptionsToPackage{hyphens}{url}\usepackage{hyperref}
\usepackage{breakurl}

% To put footnotes at the bottom of the page
\usepackage[bottom]{footmisc}

% No indent
\setlength\parindent{0pt}

% To enumerate subequations with arabic numbers (e.g. 1.1, 1.2, ecc)
\patchcmd{\subequations}{\def\theequation{\theparentequation\alph{equation}}}{\def\theequation{\theparentequation.\arabic{equation}}}{}{}

\DeclarePairedDelimiter\abs{\lvert}{\rvert}
\makeatletter
\let\oldabs\abs
\def\abs{\@ifstar{\oldabs}{\oldabs*}}

% Definition of definition environment
\theoremstyle{definition}
\newtheorem{defn}{Definition}[section]

% Definition of theorem environment
\theoremstyle{theorem}
\newtheorem{thm}{Theorem}[section]

% To enumerate the equations and the figures according to the section they are in
\numberwithin{equation}{section}
\numberwithin{figure}{section}

% Path to images folder
\graphicspath{{./img/}}

% To modify the space between figure and caption
%\setlength{\abovecaptionskip}{-4pt}
%\setlength{\belowcaptionskip}{3pt}

\renewcommand{\textfraction}{0.1}
\renewcommand{\topfraction}{0.9}

\usepackage{tikz}

\begin{document}

	\section{Introduction}

		This is a schematic description of the C++ code performing mesh reduction for a non-planar spatial regression analysis involving a penalty term. \\
		The document is structured in the following way. In Section \ref{section:Repo structure} we illustrate the contents of each folder of the Git repository \texttt{mesh-simplification}. Then in section \ref{section:} we present the interface of the main classes and functions defined in the code. A brief description for each code entity is given. To help the reader go through the rather complicated structure of the library, hyperlinks among different sections will be provided.
	
	\section{Repo structure}
	\label{section:Repo structure}
	
		The code has been organized in the following folders:
		\begin{itemize}
			\item[\texttt{/bin}] folder storing the binaries, i.e. \texttt{cpp} files with \texttt{main};
			\item[\texttt{/doc}] heterogeneous documentation regarding the code;
			\item[\texttt{/mesh}] set of test grids, useful for numerical simulations;
			\item[\texttt{/RobustPredicated}] suite of geometric standard tests for a grid; they are said \emph{robust} because they are designed to properly work even with numerically dangerous objects, e.g. badly-conditioned matrices;
			\item[\texttt{/src}] the header and source files constituing the library;
			\item[\texttt{/unitTests}] suite of short tests aiming at validating the code.
		\end{itemize}
		Within the folder \texttt{/src}, one may find the following subfolders:
		\begin{itemize}
			\item[\texttt{/core}] definition of base geometric entities, e.g. point;
			\item[\texttt{/file}] to manage I/O; note that post-processing is carried out in ParaView;
			\item[\texttt{/intersec}] check if two geometric elements, e.g. two triangles, intersect;
			\item[\texttt{/utility}] various geometric tests, e.g. check if a point falls within a polygon;
			\item[\texttt{/geometry}] definition of the data structured to store a grid; the main files it contains are:
			\begin{itemize}
				\item \texttt{connect2D.hpp}: it implements the connections between elements; remember that an element is characterized by:
				\begin{enumerate}
					\item ID,
					\item point(s),
					\item color, representing the material;
				\end{enumerate}
				\item \texttt{mesh2D.hpp}: data structure to store a two-dimensional grid;
				\item \texttt{tricky2D.hpp}: performs different operations on the grid \emph{without} altering the node-element connections;
			\end{itemize}
			\item[\texttt{/doctor}] perform the following operations on the grid:
			\begin{itemize}
				\item smoothing,
				\item edge swapping,
				\item edge splitting,
				\item edge collapsing.
			\end{itemize} 
			Note that this operations modify the node-element connections;
			\item[\texttt{/meshOperation}] compute the cost functional for each edge, whose making the \texttt{doctor} aware of which edges have to contract; the main file is \texttt{simplification2d}.
		\end{itemize}

\end{document}
